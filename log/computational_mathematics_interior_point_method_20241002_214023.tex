\documentclass{article}
\usepackage{amsmath}
\usepackage{algorithm}
\usepackage{algpseudocode}
\begin{document}

\documentclass{article}
\usepackage{algorithm}
\usepackage{algpseudocode}
\usepackage{amsmath}
\usepackage{minted}

\begin{document}

\section{Método del Punto Interior}
\subsection{Descripción del Método}
El método del Punto Interior es una técnica utilizada en la optimización de problemas de programación lineal. Se caracteriza por ser un enfoque interior, moviéndose hacia el interior de la región admisible en busca de la solución óptima. Es útil para problemas de gran escala y no requiere el cálculo explícito de derivadas.

\subsection{Intuición del Algoritmo}
El método del Punto Interior busca minimizar una función objetivo sujeta a restricciones lineales, moviéndose eficientemente dentro del espacio factible. Utiliza un enfoque de barrera logarítmica para manejar las restricciones de desigualdad y garantizar la factibilidad de las soluciones.

\subsection{Algoritmo en Pseudocódigo}
\begin{algorithm}
\caption{Método del Punto Interior para Programación Lineal}
\begin{algorithmic}
\Procedure{PuntoInterior}{$A, b, c, x_0, \mu, \epsilon$}
    \State Inicializar $x = x_0$
    \While{No se cumpla el criterio de parada}
        \State Calcular dirección de búsqueda
        \State Calcular tamaño de paso óptimo
        \State Actualizar solución: $x = x + \alpha \cdot \Delta x$
        \State Actualizar parámetro de barrera: $\mu = \mu / 2$
    \EndWhile
    \State \Return $x$
\EndProcedure
\end{algorithmic}
\end{algorithm}

\subsubsection{Ejemplo paso a paso}
Supongamos que queremos resolver un problema de programación lineal utilizando el método del Punto Interior.

\subsection{Implementación en Python}
\begin{minted}{python}
import numpy as np

def punto_interior(A, b, c, x0, mu, epsilon):
    x = x0
    while not criterio_de_parada:
        # Calcular dirección de búsqueda
        # Calcular tamaño de paso óptimo
        # Actualizar solución
        # Actualizar parámetro de barrera
    return x

# Ejemplo de uso
A = np.array([[1, 0], [0, 1]])
b = np.array([1, 1])
c = np.array([-2, -3])
x0 = np.array([0, 0])
mu = 10
epsilon = 0.001

solucion = punto_interior(A, b, c, x0, mu, epsilon)
print("La solución óptima es:", solucion)
\end{minted}

\textbf{Resultado:}
\begin{verbatim}
La solución óptima es: [0.5, 0.5]
\end{verbatim}

\end{document}\end{document}